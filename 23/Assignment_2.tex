\let\negmedspace\undefined
\let\negthickspace\undefined
\documentclass[journal,12pt]{IEEEtran}


\usepackage{cite}
\usepackage{amsmath,amssymb,amsfonts,amsthm}
\usepackage{algorithmic}
\usepackage{graphicx}
\usepackage{textcomp}
\usepackage{xcolor}
\usepackage{txfonts}
\usepackage{listings}
\usepackage{enumitem}
\usepackage{mathtools}
\usepackage{gensymb}
\usepackage[breaklinks=true]{hyperref}
\usepackage{tkz-euclide} % loads  TikZ and tkz-base
\usepackage{listings}
\usepackage{caption}

\DeclareMathOperator*{\Res}{Res}
%\renewcommand{\baselinestretch}{2}
\renewcommand\thesection{\arabic{section}}
\renewcommand\thesubsection{\thesection.\arabic{subsection}}
\renewcommand\thesubsubsection{\thesubsection.\arabic{subsubsection}}

\renewcommand\thesectiondis{\arabic{section}}
\renewcommand\thesubsectiondis{\thesectiondis.\arabic{subsection}}
\renewcommand\thesubsubsectiondis{\thesubsectiondis.\arabic{subsubsection}}

% correct bad hyphenation here
\hyphenation{op-tical net-works semi-conduc-tor}
\def\inputGnumericTable{}                                 %%

\lstset{
%language=C,
frame=single, 
breaklines=true,
columns=fullflexible
}
%\lstset{
%language=tex,
%frame=single, 
%breaklines=true
%}

\begin{document}
%


\newtheorem{theorem}{Theorem}[section]
\newtheorem{problem}{Problem}
\newtheorem{proposition}{Proposition}[section]
\newtheorem{lemma}{Lemma}[section]
\newtheorem{corollary}[theorem]{Corollary}
\newtheorem{example}{Example}[section]
\newtheorem{definition}[problem]{Definition}
%\newtheorem{thm}{Theorem}[section] 
%\newtheorem{defn}[thm]{Definition}
%\newtheorem{algorithm}{Algorithm}[section]
%\newtheorem{cor}{Corollary}
\newcommand{\BEQA}{\begin{eqnarray}}
\newcommand{\EEQA}{\end{eqnarray}}
\newcommand{\define}{\stackrel{\triangle}{=}}
\newcommand{\permcomb}[4][0mu]{{{}^{#3}\mkern#1#2_{#4}}}
\newcommand{\comb}[1][-1mu]{\permcomb[#1]{C}}
\newcommand\tab[1][1cm]{\hspace*{#1}}
\renewcommand{\abstractname}{Question}
\bibliographystyle{IEEEtran}
%\bibliographystyle{ieeetr}


\providecommand{\mbf}{\mathbf}
\providecommand{\pr}[1]{\ensuremath{\Pr\left(#1\right)}}
\providecommand{\qfunc}[1]{\ensuremath{Q\left(#1\right)}}
\providecommand{\sbrak}[1]{\ensuremath{{}\left[#1\right]}}
\providecommand{\lsbrak}[1]{\ensuremath{{}\left[#1\right.}}
\providecommand{\rsbrak}[1]{\ensuremath{{}\left.#1\right]}}
\providecommand{\brak}[1]{\ensuremath{\left(#1\right)}}
\providecommand{\lbrak}[1]{\ensuremath{\left(#1\right.}}
\providecommand{\rbrak}[1]{\ensuremath{\left.#1\right)}}
\providecommand{\cbrak}[1]{\ensuremath{\left\{#1\right\}}}
\providecommand{\lcbrak}[1]{\ensuremath{\left\{#1\right.}}
\providecommand{\rcbrak}[1]{\ensuremath{\left.#1\right\}}}
\theoremstyle{remark}
\newtheorem{rem}{Remark}
\newcommand{\sgn}{\mathop{\mathrm{sgn}}}
\providecommand{\abs}[1]{\left\vert#1\right\vert}
\providecommand{\res}[1]{\Res\displaylimits_{#1}} 
\providecommand{\norm}[1]{\left\lVert#1\right\rVert}
%\providecommand{\norm}[1]{\lVert#1\rVert}
\providecommand{\mtx}[1]{\mathbf{#1}}
\providecommand{\mean}[1]{E\left[ #1 \right]}
\providecommand{\fourier}{\overset{\mathcal{F}}{ \rightleftharpoons}}
%\providecommand{\hilbert}{\overset{\mathcal{H}}{ \rightleftharpoons}}
\providecommand{\system}{\overset{\mathcal{H}}{ \longleftrightarrow}}
	%\newcommand{\solution}[2]{\textbf{Solution:}{#1}}
\newcommand{\solution}{\noindent \textbf{Solution: }}
\newcommand{\cosec}{\,\text{cosec}\,}
\providecommand{\dec}[2]{\ensuremath{\overset{#1}{\underset{#2}{\gtrless}}}}
\newcommand{\myvec}[1]{\ensuremath{\begin{pmatrix}#1\end{pmatrix}}}
\newcommand{\mydet}[1]{\ensuremath{\begin{vmatrix}#1\end{vmatrix}}}

\let\vec\mathbf

\vspace{3cm}

\title{
\textbf {Assignment 2}\\ \large \textbf{AI1110}: Probability and Random Variables\\Indian Institute of Techonology Hyderabad
}
\author{S Bhavya Shloka\\CS22BTECH11056}
	

\maketitle

\newpage

%\tableofcontents

\bigskip

\renewcommand{\thefigure}{\theenumi}
\renewcommand{\thetable}{\theenumi}


\textbf{12.13.6.3: Question}. A game consists of tossing a one rupee coin 3 times and noting its outcome each time. Hanif wins if all the tosses give the same result i.e, three heads or three tails, and loses otherwise. Calculate the probablity that Hanif will lose the game.

\textbf{Answer: $\frac{3}{4}$.}
\\\\
\textbf{Solution}:
Let us consider a ran-dom variable X which is the number of heads occured in 3 tosses.
$X$ can take values $0,1,2,3$.

\begin{table}[h]
\centering
\resizebox{0.5\textwidth}{!}{%
\begin{tabular}{|l|l|}
\hline
Random Variable(X) & Sample Space                                              \\ \hline
0                  & TTT                                                       \\ \hline
1                  & \begin{tabular}[c]{@{}l@{}}TTH\\ THT\\ HTT\end{tabular}   \\ \hline
2                  & \begin{tabular}[c]{@{}l@{}}HHT\\ HTH\\ THH\end{tabular}   \\ \hline
3                  & HHH                                                       \\ \hline
\end{tabular}%
}
\end{table}

\begin{align}
   \pr{X=i}=
      \begin{cases}
            \frac{1}{8} &  i=0  \\
            \comb{3}{1}\times\frac{1}{8} &  i=1  \\
            \comb{3}{2}\times\frac{1}{8} &  i=2  \\
            \frac{1}{8} &  i=3
      \end{cases}
  \label{eq:PX}    
\end{align}
\\
The random variable follows binomial distribution and can be generalised into 
\begin{align}
  \pr{X=i}=\comb{3}{i}\times\frac{1}{8}
\end{align}
\\
Hanif wins the game when X=0 or X=3
So, the probability that he wins the game is $\pr{X=0 + X=3}$.\\

As X=0 and X=3 are two mutually exclusive events, $\pr{X=0,X=3}=0$

\begin{align}
  \pr{X=0 + X=3}=\pr{X=0}+\pr{X=3}\nonumber\\
 \label{eq:Win}
\end{align}

Substituting \eqref{eq:PX} in \eqref{eq:Win}, 

\begin{align}
  \pr{X=0 + X=3}&=\frac{1}{8}+\frac{1}{8}\nonumber\\
  \pr{X=0 + X=3}&=\frac{1}{4}\nonumber\\ 
\end{align}

The probability of Hanif losing the game 
\begin{align}
   &=1-\pr{X=0 + X=3}\nonumber\\
   &=1-\frac{1}{4}\nonumber\\
   &=\frac{3}{4}\nonumber\\
\end{align}

Therefore the probability that Hanif loses the game if $\frac{3}{4}$.

\end{document}